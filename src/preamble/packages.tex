% Default preamble
\usepackage[utf8]{inputenc}
\usepackage[english]{babel}
\usepackage[T1]{fontenc}
\usepackage{amsmath}
\usepackage{parskip}
\usepackage[numbers,sort&compress]{natbib}
\usepackage[colorlinks=true, citecolor=gray]{hyperref}
% xcolor isn't a default, but if using with options like `dvipsnames',
% this clashes with tikz' own import of xcolor.
% So must import it here first.
\usepackage[dvipsnames,table]{xcolor}
\usepackage{tikz}
\usepackage{lipsum}

% Custom stuff

% Inuput .tikzdefs ASAP - some packages are installed there.
\input{preamble/fockedup.tikzdefs}
\input{preamble/fockedup.tikzstyles}

% For inverse search with PDF viewer
\usepackage{pdfsync}
%% Various AMS packages - MOVED TO .tikzdefs!
%\usepackage{amsfonts}
\usepackage{amsthm}
%\usepackage{amssymb}
% For tikzit, duh.
\usepackage{preamble/tikzit}
% For making 'example' environment non-italics.
\usepackage{thmtools}
% For blackboard font style '1'.
\usepackage{dsfont}
% For braket stuff, duh.
\usepackage{braket}
%% The \coloneqq symbols, amongst others - MOVED TO .tikzdefs!
%\usepackage{mathtools}
% For using diagrams as mathematical objects, mostly.
\usepackage[export]{adjustbox}
% For multi-row (>3) group presentations.
\usepackage{scalerel}
% For including object type in references.
\usepackage[nameinlink]{cleveref}
% For defining custom 'clevercite' commands in definitions.tex.
\usepackage{xstring}
% For defining commands with optional arguments.
\usepackage{ifthen}
% For more complex subfigure stuff.
\usepackage[font=small,labelfont=bf]{caption}
\usepackage{subcaption}
% For strikethrough text via \sout.
% Adding option 'normalem' stops package from overwriting the \emph command.
\usepackage[normalem]{ulem}