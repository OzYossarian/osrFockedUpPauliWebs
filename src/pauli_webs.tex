\section{Pauli webs}

\begin{itemize}
    \item Define Pauli webs for individual spiders.
        \begin{itemize}
            \item Get scalars right
            \item Pick convention for whether position or momentum displacements are `inner' or `outer' gates.
            \item Do webs for GKP spiders.
        \end{itemize}
    \item Define Pauli webs for diagrams with multiple spiders.
\end{itemize}

Fourier transforms:
\begin{equation}\label{eq:fourier_transforms_defn}
    \begin{aligned}
        \mathcal{F}(f)(p) &= \int_{-\infty}^{\infty} f(t) e^{-i 2 \pi p t} d t \\[5pt]
        \mathcal{F}^{-1}(f)(q) &= \int_{-\infty}^{\infty} f(t) e^{i 2 \pi q t} d t
    \end{aligned}
\end{equation}

Dirac delta properties, presumably ill-defined, worry later:
\begin{equation}\label{eq:dirac_delta_properties}
    \int_{-\infty}^{\infty} e^{2 \pi i t a} dt = \delta(a)
    \qqqquad
    \int_{-\infty}^{\infty} f(t) \delta(t-a) dt = f(a)
    \qqqquad
    \delta(t) = \delta(-t)
\end{equation}

Function $\chi_x : \mathbb{R} \to \mathbb{C}$ defined as $\chi_x(p) = e^{-i2\pi xp}$.
Hence $\chi_0(p) = e^{-i2\pi \cdot 0 \cdot p} = 1$ is the constant function $1$.

No $n \in \mathbb{N}$ such that the Kronecker delta $\delta_n: \mathbb{N} \to \mathbb{C}$ is the constant function $1$.

Throughout, we will use a dot/underline/something to denote that a variable is the single argument of a function.
For example, if writing $e^{-i2\pi \dot{x}p}$, we mean the function $x \mapsto e^{-i2\pi xp}$.
If instead we write $e^{-i2\pi x\dot{p}}$, we mean the function $p \mapsto e^{-i2\pi xp}$.

\begin{lemma}
    For any $f: \mathbb{R} \to \mathbb{C}$, we have $\mathcal{F}^{-1}(\mathcal{F}(f) \cdot \chi_a) = f(x - a)$.
    \begin{proof}
        Expand and simplify.
        Hard part is keeping track of the three variables $x$, $y$ and $z$ we introduce.
        \begin{align*}
            \mathcal{F}^{-1}(\mathcal{F}(f) \cdot \chi_a)
            &= \mathcal{F}^{-1}(\mathcal{F}(f(z))(y) \cdot \chi_a(y))(x) \\
            &= \mathcal{F}^{-1}\left(\left[\int_{-\infty}^{\infty} f(z) e^{-i2\pi yz} \, dz\right] \cdot e^{-i2\pi ay}\right)(x) \\
            &= \int_{-\infty}^{\infty} \left[\int_{-\infty}^{\infty} f(z) e^{-i2\pi yz} \, dz\right] \cdot e^{-i2\pi ay} \cdot e^{i2\pi yx} \, dy \\
            &= \int_{-\infty}^{\infty} \left[\int_{-\infty}^{\infty} f(z) e^{-i2\pi yz} \, dz\right] \cdot e^{i2\pi y(x-a)} \, dy \\
            &= \int_{-\infty}^{\infty} \int_{-\infty}^{\infty} f(z) e^{i2\pi y(x-a-z)} \, dz\, dy \\
            &= \int_{-\infty}^{\infty} f(z) \left[\int_{-\infty}^{\infty} e^{i2\pi y(x-a-z)} \, dy\right] dz \\
            \intertext{Now use the Dirac delta properties from \Cref{eq:dirac_delta_properties}:}
            &= \int_{-\infty}^{\infty} f(z) \delta(x-a-z) \, dz \\
            &= \int_{-\infty}^{\infty} f(z) \delta(z - (x-a)) \, dz \\
            &= f(x-a)
        \end{align*}
    \end{proof}
\end{lemma}

\begin{lemma}\label{lem:red_spider_not_flexsymmetric}
    \begin{equation}\label{eq:red_spider_not_flexsymmetric}
        \tikzfig{red_spider_not_flexsymmetric_v2}
    \end{equation}
\end{lemma}

Need to show the copy rule extends to gates, not just states.
Presumably add some new rule(s)!?
Use integral nonsense above for soundness.

\begin{equation}\label{eq:gate_copy}
    \tikzfig{red_gate_copy}
    \qqqqquad
    \tikzfig{green_gate_copy}
\end{equation}

\begin{lemma}\label{lem:red_gate_copy_multi_inputs}
    \begin{equation}\label{eq:red_gate_copy_multi_inputs}
        \tikzfig{red_gate_copy_multi_input}
    \end{equation}
\end{lemma}

\begin{lemma}\label{lem:red_gate_copy_cancel}
    \begin{equation}\label{eq:red_gate_copy_cancel}
        \tikzfig{red_gate_copy_cancel}
    \end{equation}
\end{lemma}

\begin{definition}[Pauli webs]\label{defn:pauli_webs}
    \begin{equation}\label{eq:general_pauli_web_green_spider}
        \tikzfig{general_pauli_web_green_spider}
    \end{equation}
    \vspace{20pt}
    \begin{equation}\label{eq:general_pauli_web_red_spider}
        \tikzfig{general_pauli_web_red_spider}
    \end{equation}
\end{definition}

Note $\prod_{j} \overline{\chi}_{p_j} = \prod_{j} e^{i2\pi p_j \isarg{x}} = \exp(i2\pi (\sum_j p_j) \isarg{x})$.
This equals the constant function $1$ iff $\sum_j p_j = 0$.


