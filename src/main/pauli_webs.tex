\section{Pauli webs}

\begin{itemize}
    \item Define Pauli webs for individual spiders.
        \begin{itemize}
            \item Get scalars right
            \item Pick convention for whether position or momentum displacements are `inner' or `outer' gates.
            \item Do webs for GKP spiders.
        \end{itemize}
    \item Define Pauli webs for diagrams with multiple spiders.
\end{itemize}

Function $\chi_x : \mathbb{R} \to \mathbb{C}$ defined as $\chi_x(p) = e^{-i2\pi xp}$.
Hence $\chi_0(p) = e^{-i2\pi \cdot 0 \cdot p} = 1$ is the constant function $1$.

No $n \in \mathbb{N}$ such that the Kronecker delta $\delta_n: \mathbb{N} \to \mathbb{C}$ is the constant function $1$.

Throughout, we will use a dot/underline/something to denote that a variable is the single argument of a function.
For example, if writing $e^{-i2\pi \dot{x}p}$, we mean the function $x \mapsto e^{-i2\pi xp}$.
If instead we write $e^{-i2\pi x\dot{p}}$, we mean the function $p \mapsto e^{-i2\pi xp}$.

Will use shorthand:
\begin{equation}\label{eq:mult_default}
    \tikzfig{right_mult_default}
    \qqqqqquad
    \tikzfig{left_mult_default}
\end{equation}

Recall also that:
\begin{equation}\label{eq:mult_inverse_mult_fourier}
    \tikzfig{mult_inverse}
    \qqqqqquad
    \tikzfig{mult_fourier}
\end{equation}

Hence:
\begin{equation}\label{eq:mult_default_with_fourier}
    \tikzfig{mult_default_with_fourier}
\end{equation}


\begin{lemma}\label{lem:gate_absorb}
    \begin{equation}\label{eq:gate_absorb}
        \tikzfig{red_gate_absorb}
        \qqqqquad
        \tikzfig{green_gate_absorb}
    \end{equation}
    \begin{proof}
        First equation makes use of the fact that $\mathcal{F}^{-1}(\mathcal{F}(f(x)) \cdot \chi_a) = f(x - a)$:
        \begin{equation}\label{eq:red_gate_absorb_proof}
            \tikzfig{red_gate_absorb_proof}
        \end{equation}
        Second equation makes use of the dual property $\mathcal{F}(\mathcal{F}^{-1}(f(x)) \cdot \overline{\chi}_a) = f(x - a)$:
        \begin{equation}\label{eq:green_gate_absorb_proof}
            \tikzfig{green_gate_absorb_proof}
        \end{equation}
    \end{proof}
\end{lemma}

\begin{lemma}\label{lem:gate_copy_blank_spider}
    \begin{equation}\label{eq:gate_copy_blank_spider}
        \tikzfig{red_gate_copy_blank_spider}
        \qqqqquad
        \tikzfig{green_gate_copy_blank_spider}
    \end{equation}
    \begin{proof}
        Follows from state copy and bialgebra rules:
        \begin{equation}\label{eq:red_gate_copy_blank_spider_proof}
            \tikzfig{red_gate_copy_blank_spider_proof}
        \end{equation}
    \end{proof}
\end{lemma}

\begin{lemma}\label{lem:red_spider_not_flexsymmetric}
    \begin{equation}\label{eq:red_spider_not_flexsymmetric}
        \tikzfig{red_spider_not_flexsymmetric_v2}
    \end{equation}
\end{lemma}

\begin{lemma}\label{lem:gate_copy}
    \begin{equation}\label{eq:gate_copy}
        \tikzfig{red_gate_copy}
        \qqqqquad
        \tikzfig{green_gate_copy}
    \end{equation}
    \begin{proof}
        \begin{equation}\label{eq:red_gate_copy_proof}
            \tikzfig{red_gate_copy_proof}
        \end{equation}
    \end{proof}
\end{lemma}

\begin{lemma}\label{lem:red_gate_copy_multi_inputs}
    \begin{equation}\label{eq:red_gate_copy_multi_inputs}
        \tikzfig{red_gate_copy_multi_input}
    \end{equation}
\end{lemma}

\begin{lemma}\label{lem:chi_constant_iff}
    The function $\chi_a(x) = e^{-i2\pi ax}$ is a constant function iff $a = 0$, in which case it's the constant function $1$.
\end{lemma}

\begin{lemma}\label{lem:red_gate_copy_cancel}
    \begin{equation}\label{eq:red_gate_copy_cancel}
        \tikzfig{red_gate_copy_cancel}
    \end{equation}
\end{lemma}

%TODO - maybe replace spiders in the middle with very generic diagram boxes?
TODO - change this - in OG paper the generators are the $Z$ and Fock spiders, so maybe stick to that here?
And give the red spider webs as a special case of webs on diagrams with multiple spiders?
\begin{definition}[Pauli web on a single $Z$ or $X$ spider]\label{defn:pauli_webs}
    \begin{equation}\label{eq:general_pauli_web_green_spider}
        \tikzfig{general_pauli_web_green_spider}
    \end{equation}
    \vspace{20pt}
    \begin{equation}\label{eq:general_pauli_web_red_spider}
        \tikzfig{general_pauli_web_red_spider}
    \end{equation}
\end{definition}

\begin{proposition}\label{thm:general_pauli_web_green_gaussian_spider}
    A highlighting of edges around a Gaussian $Z$ spider with phase function $e^{i(t + u\isarg{x} + v\isarg{x}^2)}$ is a Pauli web iff it's of the following form, where $2\pi \sum_j a_j = 2vb$:
    \begin{equation}\label{eq:general_pauli_web_green_linear_spider}
        \tikzfig{general_pauli_web_green_gaussian_spider}
    \end{equation}
    \begin{proof}
        TODO
    \end{proof}
\end{proposition}

\begin{definition}\label{defn:indicator_function}
    For any predicate $f: \mathbb{R} \to \mathbb{B}$, we define $\indicator{f(\isarg{x})} : \mathbb{R} \to \mathbb{C}$ as:
    \begin{equation}\label{eq:indicator_function}
        \indicator{f(x)} = \begin{cases}
            1 &\text{ if } f(x) \\
            0 &\text{ else }
        \end{cases}
    \end{equation}
\end{definition}

We will also use the shorthand $x \eqmod{r} y$ to mean $x = y$ mod $r$.

\begin{proposition}\label{thm:general_pauli_web_green_gkp_like_spider}
A highlighting of edges around a $Z$ spider with phase function $\llbracket \isarg{x} \eqmod{r} 0 \rrbracket$ is a Pauli web iff it's of the following form, where $b \eqmod{r} 0$ and $r \sum_j a_j \eqmod{1} 0$:
\begin{equation}\label{eq:general_pauli_web_green_gkp_like_spider}
    \tikzfig{general_pauli_web_green_gkp_like_spider}
\end{equation}
\begin{proof}
    TODO
\end{proof}
\end{proposition}
