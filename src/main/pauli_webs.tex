\section{Pauli webs}\label{sec:pauli-webs}

Function $\chi_x : \mathbb{R} \to \mathbb{C}$ defined as $\chi_x(p) = e^{-i2\pi xp}$.
Note $\chi_0(p) = e^{-i2\pi \cdot 0 \cdot p} = 1$ is the constant function $1$.
Indeed, the function $\chi_a(x) = e^{-i2\pi ax}$ is a constant function iff $a = 0$.
No $n \in \mathbb{N}$ such that the Kronecker delta $\delta_n: \mathbb{N} \to \mathbb{C}$ is the constant function $1$.

Throughout, we will use a dot/underline/something to denote that a variable is the single argument of a function.
For example, if writing $e^{-i2\pi \dot{x}p}$, we mean the function $x \mapsto e^{-i2\pi xp}$.
If instead we write $e^{-i2\pi x\dot{p}}$, we mean the function $p \mapsto e^{-i2\pi xp}$.

\subsection{Defining Pauli webs}\label{subsec:defining-pauli-webs}

\begin{definition}\label{defn:generators}
    The \textbf{generators} of the Focked-up ZX-calculus are the three spiders:
    \begin{equation}\label{eq:generators_spiders}
        \tikzfig{generators_spiders}
    \end{equation}
    the multiplier and $W$-node:
    \begin{equation}\label{eq:generators_mult_W_node}
    \tikzfig{generators_mult_W_node}
    \end{equation}
    and the identity and swap maps:
    \begin{equation}\label{eq:generators_id_swap}
    \tikzfig{generators_id_swap}
    \end{equation}
    where $f, h: \mathbb{R} \to \mathbb{C}$, $g: \mathbb{N} \to \mathbb{C}$ and $r \in \mathbb{R}$.
\end{definition}

In what follows, it might help to temporarily think of the identity and swap maps as also having a node at their center:
\begin{equation}\label{eq:generators_id_swap_explicit}
    \tikzfig{generators_id_swap_explicit}
\end{equation}

Note that these are an overcomplete generating set; will discuss this shortly!
Will draw a generic generator as an asymmetric box with some label $f$ and some number of input and output legs:
\begin{equation}\label{eq:generic_generator}
    \tikzfig{generator}
\end{equation}

\begin{definition}\label{defn:highlighting_generator}
    A \textbf{highlighting} of a single generator is an assignment of a tuple $(a_j, b_j) \in \mathbb{R}^2$ to each of its input and output legs.
    Graphically:
    \begin{equation}\label{eq:highlighting_generator}
        \tikzfig{highlighting_generator}
    \end{equation}
    Whenever one or both of $a_j, b_j$ are zero, we can omit the corresponding highlight from the diagram:
    \begin{equation}\label{eq:highlighting_generator_0s_omitted}
        \tikzfig{highlighting_generator_0s_omitted}
    \end{equation}
\end{definition}

\begin{definition}\label{defn:pauli_web_generator}
    A \textbf{Pauli web} on a single generator is a highlighting:
    \begin{equation}\label{eq:pauli_web_generator_highlighting}
        \tikzfig{highlighting_generator}
    \end{equation}
    satisfying:
    \begin{equation}\label{eq:pauli_web_generator_condition}
        \tikzfig{pauli_web_generator_condition}
    \end{equation}
\end{definition}

\begin{definition}\label{defn:boundary_internal_edges}
    We say an edge in any ZX-diagram is a \textbf{boundary edge} if it is an input or output of the overall diagram, else it's an \textbf{internal edge}.
\end{definition}

Potential gotchas here are the diagram consisting of just the identity map, and the diagram consisting of just the swap map.
The former has two boundary edges, not one, and similarly the latter has four boundary edges, not two:
\begin{equation}\label{eq:generators_id_swap_explicit_boundary_edges}
    \tikzfig{generators_id_swap_explicit}
\end{equation}

\begin{definition}\label{defn:half_edges}
    We declare that internal edges actually consist of two \textbf{half-edges}, while boundary edges are a single \textbf{half-edge}.
\end{definition}

\begin{definition}\label{defn:highlighting_general_diagram}
    A \textbf{highlighting} of a general ZX-diagram is an assignment of a tuple in $\mathbb{R}^2$ to each half-edge.
    Graphically:
    \begin{equation}\label{eq:highlighting_general_diagram}
        \tikzfig{highlighting_general_diagram}
    \end{equation}
    Again, whenever one or both components of the tuple are zero, we may omit the corresponding highlight.
\end{definition}

\begin{definition}\label{defn:pauli_web_general_diagram}
    A \textbf{Pauli web} on a general ZX-diagram is a highlighting:
    \begin{equation}\label{eq:pauli_web_general_diagram_highlighting}
        \tikzfig{highlighting_general_diagram}
    \end{equation}
    such the restriction to each generator and its incident half-edges is a Pauli web for that generator, and on every internal edge the pair of tuples $(a, b)$ and $(c, d)$ satisfy $c = -a$ and $d = -b$:
    \begin{equation}\label{eq:pauli_web_general_diagram_conditions}
        \tikzfig{pauli_web_general_diagram}
    \end{equation}
\end{definition}

TODO: what to say about Pauli webs being somehow preserved under rewrites?
i.e.\ suppose there's a Pauli web on a diagram $D$ with boundary edges labelled $(a_j, b_j)$.
If $D = D'$ for some other diagram $D$, then there should be a valid Pauli web on $D'$ with the same boundary edge labels.
This is a generalisation of my initial concern which was about consistency of these two definitions given that our set of generators is overcomplete - e.g.:
\begin{equation}\label{eq:red_spider_derived}
    \tikzfig{red_spider_derived}
\end{equation}

TODO: add intuition/explanations throughout all this

TODO: can we get away with just using one label $(a, b)$ for each internal edge?
Was thinking no - counterexample being two generators connected by a cup or cap.

\subsection{Determining Pauli webs}\label{subsec:determining-pauli-webs}

Will use shorthand:
\begin{equation}\label{eq:mult_default}
    \tikzfig{right_mult_default}
    \qqqqqquad
    \tikzfig{left_mult_default}
\end{equation}

Recall also that:
\begin{equation}\label{eq:mult_inverse_mult_fourier}
    \tikzfig{mult_inverse}
    \qqqqqquad
    \tikzfig{mult_fourier}
\end{equation}

Hence:
\begin{equation}\label{eq:mult_default_with_fourier}
    \tikzfig{mult_default_with_fourier}
\end{equation}

\begin{lemma}\label{lem:gate_absorb}
    \begin{equation}\label{eq:gate_absorb}
        \tikzfig{red_gate_absorb}
        \qqqqquad
        \tikzfig{green_gate_absorb}
    \end{equation}
    \begin{proof}
        First equation makes use of the fact that $\mathcal{F}^{-1}(\mathcal{F}(f(x)) \cdot \chi_a) = f(x - a)$:
        \begin{equation}\label{eq:red_gate_absorb_proof}
            \tikzfig{red_gate_absorb_proof}
        \end{equation}
        Second equation makes use of the dual property $\mathcal{F}(\mathcal{F}^{-1}(f(x)) \cdot \overline{\chi}_a) = f(x - a)$:
        \begin{equation}\label{eq:green_gate_absorb_proof}
            \tikzfig{green_gate_absorb_proof}
        \end{equation}
    \end{proof}
\end{lemma}

\begin{lemma}\label{lem:gate_copy_blank_spider}
    \begin{equation}\label{eq:gate_copy_blank_spider}
        \tikzfig{red_gate_copy_blank_spider}
        \qqqqquad
        \tikzfig{green_gate_copy_blank_spider}
    \end{equation}
    \begin{proof}
        Follows from state copy and bialgebra rules:
        \begin{equation}\label{eq:red_gate_copy_blank_spider_proof}
            \tikzfig{red_gate_copy_blank_spider_proof}
        \end{equation}
    \end{proof}
\end{lemma}

\begin{lemma}\label{lem:red_spider_not_flexsymmetric}
    \begin{equation}\label{eq:red_spider_not_flexsymmetric}
        \tikzfig{red_spider_not_flexsymmetric_v2}
    \end{equation}
\end{lemma}

\begin{lemma}\label{lem:gate_copy}
\begin{equation}\label{eq:gate_copy}
\tikzfig{red_gate_copy}
\qqqqquad
\tikzfig{green_gate_copy}
\end{equation}
\begin{proof}
    \begin{equation}\label{eq:red_gate_copy_proof}
        \tikzfig{red_gate_copy_proof}
    \end{equation}
\end{proof}
\end{lemma}

\begin{lemma}\label{lem:red_gate_copy_multi_inputs}
    \begin{equation}\label{eq:red_gate_copy_multi_inputs}
        \tikzfig{red_gate_copy_multi_input}
    \end{equation}
\end{lemma}

\begin{lemma}\label{lem:red_gate_copy_cancel}
    \begin{equation}\label{eq:red_gate_copy_cancel}
        \tikzfig{red_gate_copy_cancel}
    \end{equation}
\end{lemma}

\begin{proposition}\label{thm:general_pauli_web_green_gaussian_spider}
    A highlighting of a Gaussian $Z$ spider with phase function $e^{i(t + u\isarg{x} + v\isarg{x}^2)}$ is a Pauli web iff it's of the following form, where $2\pi \sum_j a_j = 2vb$:
    \begin{equation}\label{eq:general_pauli_web_green_linear_spider}
        \tikzfig{general_pauli_web_green_gaussian_spider}
    \end{equation}
    \begin{proof}
        TODO
    \end{proof}
\end{proposition}

\begin{definition}\label{defn:indicator_function}
    For any predicate $f: \mathbb{R} \to \mathbb{B}$, we define $\indicator{f(\isarg{x})} : \mathbb{R} \to \mathbb{C}$ as:
    \begin{equation}\label{eq:indicator_function}
        \indicator{f(x)} = \begin{cases}
            1 &\text{ if } f(x) \\
            0 &\text{ else }
        \end{cases}
    \end{equation}
\end{definition}

We will also use the shorthand $x \eqmod{r} y$ to mean $x = y$ mod $r$.

\begin{proposition}\label{thm:general_pauli_web_green_gkp_like_spider}
    A highlighting of a $Z$ spider with phase function $\llbracket \isarg{x} \eqmod{r} 0 \rrbracket$ is a Pauli web iff it's of the following form, where $b \eqmod{r} 0$ and $r \sum_j a_j \eqmod{1} 0$:
    \begin{equation}\label{eq:general_pauli_web_green_gkp_like_spider}
        \tikzfig{general_pauli_web_green_gkp_like_spider}
    \end{equation}
    \begin{proof}
        TODO
    \end{proof}
\end{proposition}
