% Default preamble
\pdfoutput=1
\usepackage[utf8]{inputenc}
\usepackage[english]{babel}
\usepackage[T1]{fontenc}
\usepackage{amsmath}
\usepackage{parskip}
\usepackage[numbers,sort&compress]{natbib}
\usepackage[colorlinks=true, citecolor=gray]{hyperref}
% xcolor isn't a default, but if using with options like `dvipsnames',
% this clashes with tikz' own import of xcolor.
% So must import it here first.
\usepackage[dvipsnames,table]{xcolor}
\usepackage{tikz}
\usepackage{lipsum}

% Custom stuff
% For inverse search with PDF viewer
\usepackage{pdfsync}
% Various AMS packages
\usepackage{amsfonts}
\usepackage{amsthm}
\usepackage{amssymb}
% For tikzit - color package needed for tikzdefs.
%\usepackage{color}
\usepackage{tikzit}
% For making 'example' environment non-italics.
\usepackage{thmtools}
% For blackboard font style '1'.
\usepackage{dsfont}
% For braket stuff, duh.
\usepackage{braket}
% The \coloneqq symbols, amongst others
\usepackage{mathtools}
% For using diagrams as mathematical objects, mostly.
\usepackage[export]{adjustbox}
% For multi-row (>3) group presentations.
\usepackage{scalerel}
% For including object type in references.
\usepackage[nameinlink]{cleveref}
% For defining custom 'clevercite' commands in definitions.tex.
\usepackage{xstring}
% For defining commands with optional arguments.
\usepackage{ifthen}
% For St Mary Road symbols, including double square bracket for quantum codes.
\usepackage{stmaryrd}
% For more complex subfigure stuff.
\usepackage[font=small,labelfont=bf]{caption}
\usepackage{subcaption}
% For strikethrough text via \sout.
% Adding option 'normalem' stops package from overwriting the \emph command.
\usepackage[normalem]{ulem}

% Allow figures placed at the bottom of documents to take up more space.
% The default is .3 (so figures must fit in the bottom 30% of the page)
\renewcommand\bottomfraction{.8}
% If wanting even more space, can also change topfraction (max, default .7) and textfraction (min, default .3).
\renewcommand\topfraction{.8}
\renewcommand\textfraction{.2}

% Set the depth of the table of contents
\setcounter{tocdepth}{2}