% THEOREMS
% Reset the counter for all of these every section.
% Only have one counter total - let it be the 'theorem' counter.

% Standard
\theoremstyle{plain}
\newtheorem{theorem}{Theorem}[section]
\newtheorem{plottwist}[theorem]{Plot-twist}
\newtheorem{keyidea}[theorem]{Key idea}
\newtheorem{keypoint}[theorem]{Key point}
\newtheorem{spoiler}[theorem]{Spoiler alert}
\newtheorem{definition}[theorem]{Definition}
\newtheorem{proposition}[theorem]{Proposition}
\newtheorem{lemma}[theorem]{Lemma}
\newtheorem{corollary}[theorem]{Corollary}
\newtheorem{conjecture}[theorem]{Conjecture}
\newtheorem{question}[theorem]{Question}
\newtheorem{openquestion}[theorem]{Open question}
\newtheorem{remark}[theorem]{Remark}
\newtheorem{difference}[theorem]{Difference}

% If using Cref, it needs to know how to pluralise these non-standard theorem names.
\crefname{keypoint}{key point}{key points}
\Crefname{keypoint}{Key point}{Key points}
\crefname{difference}{difference}{differences}
\Crefname{difference}{Difference}{Differences}

% Custom
\declaretheoremstyle[qed={$\blacksquare$}]{exampleStyle}
\declaretheorem[style=exampleStyle, sibling=theorem, name=Example]{example}
\declaretheorem[style=exampleStyle, sibling=theorem, name=Example, numbered=no]{example*}
% Define a new command for subsubsubsection
\makeatletter
\newcounter{subsubsubsection}[subsubsection]
\renewcommand{\thesubsubsubsection}{\thesubsubsection.\arabic{subsubsubsection}}
\newcommand{\subsubsubsection}[1]{%
    \refstepcounter{subsubsubsection}%
    \vskip 2ex \@plus 1ex \@minus .2ex% Reduced vertical space before
        {\sffamily\normalsize% Sans-serif font family, normal size, no bold
    \thesubsubsubsection \quad #1\par}% Put heading on its own line
    \nobreak% Prevent page break after heading
% \vskip 1ex \@plus .2ex% Reduced space after
}
\makeatother

% REFERENCES
% Define a `clever cite' command analogous to cleveref's functionality for references
\newcommand{\ccite}[2][]{%
    \IfSubStr{#2}{,}{refs.~}{ref.~}%
    \ifthenelse{\equal{#1}{}}{\cite{#2}}{\cite[#1]{#2}}}
\newcommand{\Ccite}[2][]{%
    \IfSubStr{#2}{,}{Refs.~}{Ref.~}%
    \ifthenelse{\equal{#1}{}}{\cite{#2}}{\cite[#1]{#2}}}
% Little QEC zoo icon with URL.
% This first one is useful as an optional argument to \cite.
\newcommand{\zoo}[1]{%
    \href{%
        https://errorcorrectionzoo.org/c/#1%
    }{\includegraphics[height=.75\baselineskip, valign=c]{%
        figures/zoo_icon}}}
% Use this one if the only thing being cited is the zoo.
\newcommand{\citeZoo}[1]{\mbox{[\zoo{#1}]}}

% CALLIGRAPHY
\newcommand{\ZZ}{\mathbb{Z}}
\newcommand{\RR}{\mathbb{R}}
\newcommand{\CC}{\mathbb{C}}

\newcommand{\DDD}{\mathcal{D}}
\newcommand{\GGG}{\mathcal{G}}
\newcommand{\HHH}{\mathcal{H}}
\newcommand{\LLL}{\mathcal{L}}
\newcommand{\MMM}{\mathcal{M}}
\newcommand{\NNN}{\mathcal{N}}
\newcommand{\PPP}{\mathcal{P}}
\newcommand{\SSS}{\mathcal{S}}
\newcommand{\TTT}{\mathcal{T}}

\newcommand{\one}{\mathds{1}}

% SYMBOLS
\newcommand{\halfuparrow}{\scalebox{0.5}{\ensuremath{\uparrow}}}
\newcommand{\halfrightarrow}{\scalebox{0.5}{\ensuremath{\rightarrow}}}

% COMMENTS
\newcommand{\teague}[1]{\comment{blue}{Teague}{#1}}

% MATHS
\newcommand{\defeq}{\coloneqq}
\newcommand{\after}{\circ}
\newcommand{\floor}[1]{\lfloor#1\rfloor}
\newcommand{\ceil}[1]{\lceil#1\rceil}
\newcommand{\nonNegInts}{\ZZ_{\geq 0}}
\newcommand{\positiveInts}{\ZZ_{>0}}
\newcommand{\groupPres}[1]{\left\langle #1 \right\rangle}

% QUANTUM
\newcommand{\allIdentitiesExcept}[1]{I \otimes \ldots \otimes I \otimes #1 \otimes I \otimes \ldots \otimes I}
\newcommand{\isg}[1]{\SSS_{#1}}
\newcommand{\lpg}[1]{\LLL_{#1}}
\newcommand{\lpglong}[1]{\NNN(\isg{#1}) / \isg{#1}}
\newcommand{\dsg}[1]{\DDD_{#1}}
\newcommand{\qcode}[1]{\llbracket#1\rrbracket}
\newcommand{\ccode}[1]{[#1]}

% LAZINESS
\newcommand{\ol}[1]{\overline{#1}}
\newcommand{\qqquad}{\quad\qquad}
\newcommand{\qqqquad}{\quad\qqquad}
\newcommand{\qqqqquad}{\quad\qqqquad}
\newcommand{\qqqqqquad}{\quad\qqqqquad}

\newcommand\blfootnote[1]{%
    \begingroup
    \renewcommand\thefootnote{}\footnote{#1}%
    \addtocounter{footnote}{-1}%
    \endgroup
}

% TIKZ
\definecolor{zx_grey}{RGB}{211,211,211}
\definecolor{zx_red}{RGB}{239, 190, 190}
\definecolor{zx_green}{RGB}{216,248,216}
